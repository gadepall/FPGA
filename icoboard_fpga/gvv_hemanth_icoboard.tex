\documentclass[journal,12pt,twocolumn]{IEEEtran}
%
\makeatletter
\@addtoreset{figure}{problem}
\makeatother
\usepackage{setspace}
\usepackage{gensymb}
\usepackage{xcolor}
\usepackage{caption}
%\usepackage{subcaption}
%\doublespacing
\singlespacing

%\usepackage{graphicx}
%\usepackage{amssymb}
%\usepackage{relsize}
\usepackage[cmex10]{amsmath}
\usepackage{mathtools}
%\usepackage{amsthm}
%\interdisplaylinepenalty=2500
%\savesymbol{iint}
%\usepackage{txfonts}
%\restoresymbol{TXF}{iint}
%\usepackage{wasysym}
\usepackage{amsthm}
\usepackage{mathrsfs}
\usepackage{txfonts}
\usepackage{stfloats}
\usepackage{cite}
\usepackage{cases}
\usepackage{subfig}
\usepackage{hyperref}
%\usepackage{xtab}
\usepackage{longtable}
\usepackage{multirow}
%\usepackage{algorithm}
%\usepackage{algpseudocode}
\usepackage{enumitem}
\usepackage{mathtools}
\usepackage{iithtlc}
%\usepackage[framemethod=tikz]{mdframed}
\usepackage{listings}
\usepackage{listings}
    \usepackage[latin1]{inputenc}                                 %%
    \usepackage{color}                                            %%
    \usepackage{array}                                            %%
    \usepackage{longtable}                                        %%
    \usepackage{calc}                                             %%
    \usepackage{multirow}                                         %%
    \usepackage{hhline}                                           %%
    \usepackage{ifthen}                                           %%
  %optionally (for landscape tables embedded in another document): %%
    \usepackage{lscape}     
\usepackage{tikz}


%\usepackage{stmaryrd}


%\usepackage{wasysym}
%\newcounter{MYtempeqncnt}
\DeclareMathOperator*{\Res}{Res}
%\renewcommand{\baselinestretch}{2}
\renewcommand\thesection{\arabic{section}}
\renewcommand\thesubsection{\thesection.\arabic{subsection}}
\renewcommand\thesubsubsection{\thesubsection.\arabic{subsubsection}}

\renewcommand\thesectiondis{\arabic{section}}
\renewcommand\thesubsectiondis{\thesectiondis.\arabic{subsection}}
\renewcommand\thesubsubsectiondis{\thesubsectiondis.\arabic{subsubsection}}

% correct bad hyphenation here
\hyphenation{op-tical net-works semi-conduc-tor}

%\lstset{
%language=C,
%frame=single, 
%breaklines=true
%}

%\lstset{
	%%basicstyle=\small\ttfamily\bfseries,
	%%numberstyle=\small\ttfamily,
	%language=Octave,
	%backgroundcolor=\color{white},
	%%frame=single,
	%%keywordstyle=\bfseries,
	%%breaklines=true,
	%%showstringspaces=false,
	%%xleftmargin=-10mm,
	%%aboveskip=-1mm,
	%%belowskip=0mm
%}

%\surroundwithmdframed[width=\columnwidth]{lstlisting}
\def\inputGnumericTable{}                                 %%

\lstset{
language=C,
frame=single, 
breaklines=true
}
 

\begin{document}
%

\theoremstyle{definition}
\newtheorem{theorem}{Theorem}[section]
\newtheorem{problem}{Problem}
\newtheorem{proposition}{Proposition}[section]
\newtheorem{lemma}{Lemma}[section]
\newtheorem{corollary}[theorem]{Corollary}
\newtheorem{example}{Example}[section]
\newtheorem{definition}{Definition}[section]
%\newtheorem{algorithm}{Algorithm}[section]
%\newtheorem{cor}{Corollary}
\newcommand{\BEQA}{\begin{eqnarray}}
\newcommand{\EEQA}{\end{eqnarray}}
\newcommand{\define}{\stackrel{\triangle}{=}}

\bibliographystyle{IEEEtran}
%\bibliographystyle{ieeetr}

\providecommand{\nCr}[2]{\,^{#1}C_{#2}} % nCr
\providecommand{\nPr}[2]{\,^{#1}P_{#2}} % nPr
\providecommand{\mbf}{\mathbf}
\providecommand{\pr}[1]{\ensuremath{\Pr\left(#1\right)}}
\providecommand{\qfunc}[1]{\ensuremath{Q\left(#1\right)}}
\providecommand{\sbrak}[1]{\ensuremath{{}\left[#1\right]}}
\providecommand{\lsbrak}[1]{\ensuremath{{}\left[#1\right.}}
\providecommand{\rsbrak}[1]{\ensuremath{{}\left.#1\right]}}
\providecommand{\brak}[1]{\ensuremath{\left(#1\right)}}
\providecommand{\lbrak}[1]{\ensuremath{\left(#1\right.}}
\providecommand{\rbrak}[1]{\ensuremath{\left.#1\right)}}
\providecommand{\cbrak}[1]{\ensuremath{\left\{#1\right\}}}
\providecommand{\lcbrak}[1]{\ensuremath{\left\{#1\right.}}
\providecommand{\rcbrak}[1]{\ensuremath{\left.#1\right\}}}
\theoremstyle{remark}
\newtheorem{rem}{Remark}
\newcommand{\sgn}{\mathop{\mathrm{sgn}}}
\providecommand{\abs}[1]{\left\vert#1\right\vert}
\providecommand{\res}[1]{\Res\displaylimits_{#1}} 
\providecommand{\norm}[1]{\lVert#1\rVert}
\providecommand{\mtx}[1]{\mathbf{#1}}
\providecommand{\mean}[1]{E\left[ #1 \right]}
\providecommand{\fourier}{\overset{\mathcal{F}}{ \rightleftharpoons}}
%\providecommand{\hilbert}{\overset{\mathcal{H}}{ \rightleftharpoons}}
\providecommand{\system}{\overset{\mathcal{H}}{ \longleftrightarrow}}
	%\newcommand{\solution}[2]{\textbf{Solution:}{#1}}
\newcommand{\solution}{\noindent \textbf{Solution: }}
\providecommand{\dec}[2]{\ensuremath{\overset{#1}{\underset{#2}{\gtrless}}}}
%\numberwithin{equation}{subsection}
\numberwithin{equation}{problem}
%\numberwithin{problem}{subsection}
%\numberwithin{definition}{subsection}
%\makeatletter
%\@addtoreset{figure}{problem}
%\makeatother
%
%\let\StandardTheFigure\thefigure
%%\renewcommand{\thefigure}{\theproblem.\arabic{figure}}
%\renewcommand{\thefigure}{\theproblem}


%\numberwithin{figure}{subsection}

%\numberwithin{equation}{subsection}
%\numberwithin{equation}{section}
%%\numberwithin{equation}{problem}
%%\numberwithin{problem}{subsection}
\numberwithin{problem}{section}
%%\numberwithin{definition}{subsection}
%\makeatletter
%\@addtoreset{figure}{problem}
%\makeatother
%\makeatletter
%\@addtoreset{table}{problem}
%\makeatother
%
%\let\StandardTheFigure\thefigure
%\let\StandardTheTable\thetable
%%%\renewcommand{\thefigure}{\theproblem.\arabic{figure}}
%%\renewcommand{\thefigure}{\theproblem}
%\renewcommand{\thetable}{\theproblem}
%%\numberwithin{figure}{section}

%%\numberwithin{figure}{subsection}



\def\putbox#1#2#3{\makebox[0in][l]{\makebox[#1][l]{}\raisebox{\baselineskip}[0in][0in]{\raisebox{#2}[0in][0in]{#3}}}}
     \def\rightbox#1{\makebox[0in][r]{#1}}
     \def\centbox#1{\makebox[0in]{#1}}
     \def\topbox#1{\raisebox{-\baselineskip}[0in][0in]{#1}}
     \def\midbox#1{\raisebox{-0.5\baselineskip}[0in][0in]{#1}}

\vspace{3cm}

%\title{ 
%	\logo{
%	Arduino for Schools
%	}
%}



% paper title
% can use linebreaks \\ within to get better formatting as desired
\title{
\logo{
Introduction to Verilog Programming on Icoboard FPGA
}
}
%
%
% author names and IEEE memberships
% note positions of commas and nonbreaking spaces ( ~ ) LaTeX will not break
% a structure at a ~ so this keeps an author's name from being broken across
% two lines.
% use \thanks{} to gain access to the first footnote area
% a separate \thanks must be used for each paragraph as LaTeX2e's \thanks
% was not built to handle multiple paragraphs
%

\author{Hemanth Kumar Desineedi and G V V Sharma$^{*}$% <-this % stops a space
\thanks{*The authors are with the Department
of Electrical Engineering, Indian Institute of Technology, Hyderabad
502285 India e-mail:  gadepall@iith.ac.in. All content in this manual is released under GNU GPL.  Free and open source.}% <-this % stops a space
%\thanks{J. Doe and J. Doe are with Anonymous University.}% <-this % stops a space
%\thanks{Manuscript received April 19, 2005; revised January 11, 2007.}}
}
% note the % following the last \IEEEmembership and also \thanks - 
% these prevent an unwanted space from occurring between the last author name
% and the end of the author line. i.e., if you had this:
% 
% \author{....lastname \thanks{...} \thanks{...} }
%                     ^------------^------------^----Do not want these spaces!
%
% a space would be appended to the last name and could cause every name on that
% line to be shifted left slightly. This is one of those "LaTeX things". For
% instance, "\textbf{A} \textbf{B}" will typeset as "A B" not "AB". To get
% "AB" then you have to do: "\textbf{A}\textbf{B}"
% \thanks is no different in this regard, so shield the last } of each \thanks
% that ends a line with a % and do not let a space in before the next \thanks.
% Spaces after \IEEEmembership other than the last one are OK (and needed) as
% you are supposed to have spaces between the names. For what it is worth,
% this is a minor point as most people would not even notice if the said evil
% space somehow managed to creep in.



% The paper headers
%\markboth{Journal of \LaTeX\ Class Files,~Vol.~6, No.~1, January~2007}%
%{Shell \MakeLowercase{\textit{et al.}}: Bare Demo of IEEEtran.cls for Journals}
% The only time the second header will appear is for the odd numbered pages
% after the title page when using the twoside option.
% 
% *** Note that you probably will NOT want to include the author's ***
% *** name in the headers of peer review papers.                   ***
% You can use \ifCLASSOPTIONpeerreview for conditional compilation here if
% you desire.




% If you want to put a publisher's ID mark on the page you can do it like
% this:
%\IEEEpubid{0000--0000/00\$00.00~\copyright~2007 IEEE}
% Remember, if you use this you must call \IEEEpubidadjcol in the second
% column for its text to clear the IEEEpubid mark.



% make the title area
\maketitle

\tableofcontents

\bigskip

\begin{abstract}
%\boldmath
This manual provides an introduction to
Verilog programming using the Icoboard-Lattice FPGA. This is
done by implementing a decade counter using verilog. The
process is likely to be similar for other FPGA boards as
well.
\end{abstract}
% IEEEtran.cls defaults to using nonbold math in the Abstract.
% This preserves the distinction between vectors and scalars. However,
% if the journal you are submitting to favors bold math in the abstract,
% then you can use LaTeX's standard command \boldmath at the very start
% of the abstract to achieve this. Many IEEE journals frown on math
% in the abstract anyway.

% Note that keywords are not normally used for peerreview papers.
%\begin{IEEEkeywords}
%Cooperative diversity, decode and forward, piecewise linear
%\end{IEEEkeywords}



% For peer review papers, you can put extra information on the cover
% page as needed:
% \ifCLASSOPTIONpeerreview
% \begin{center} \bfseries EDICS Category: 3-BBND \end{center}
% \fi
%
% For peerreview papers, this IEEEtran command inserts a page break and
% creates the second title. It will be ignored for other modes.
\IEEEpeerreviewmaketitle


%\newpage
%\section{Component Pin Diagrams}
%%
%\input{chapter1}
%
\section{Componets}
\input{./figs/components.tex}
%\newpage
\section{Software Setup}
\begin{enumerate}
%\item Boot the Raspberry PI from the SD card.
%\item Open a console window and expand the root filesystem by typing the command below 
%
%\begin{verbatim}
%
%sudo raspi-config
%
%\end{verbatim}
%select the "Expand Filesystem" and reboot the Raspberry PI.

\item Open a console window and execute the following commands for installing wiringPi, IcoProg, IcoStorm tools, Arachne-pnr and Yosys.
\begin{lstlisting}

cd $HOME
git clone git://git.drogon.net/wiringPi
cd wiringPi && ./build

cd $HOME
sudo apt-get install subversion
svn co http://svn.clifford.at/handicraft/2015/icoprog
cd icoprog && make install

sudo apt-get install build-essential clang bison flex libreadline-dev 
sudo apt-get install gawk tcl-dev libffi-dev git mercurial graphviz   
sudo apt-get install xdot pkg-config python python3 libftdi-dev

cd $HOME
git clone https://github.com/cliffordwolf/icestorm.git icestorm
cd icestorm && make && sudo make install

cd $HOME
git clone https://github.com/cseed/arachne-pnr.git arachne-pnr
cd arachne-pnr && make && sudo make install

cd $HOME
git clone https://github.com/cliffordwolf/yosys.git yosys
cd yosys && make && sudo make install

\end{lstlisting}



%\item Download the raspbian image with icoTC (yosys, ArachnePnR, icestorm) and risc-v compiler toolchain. To download it click on the following link.

%\url{http://files.clifford.at/2017-03-02-raspbian-jessie-icotools.zip}

%\item Unzip the file.

%\item In Ubuntu, to write this image on a SD card, right click on it and click on "Open With Disk Image Writer".

%\item Select the SD card in "Restore Disk Image" dialog box.

%\item Click on "Start Restoring", then the image will write on SD card.

%\item For writing an image on the SD card on Windows, see the given link below for instructions:

%\url{https://www.raspberrypi.org/documentation/installation/installing-images/README.md}
\item Open a text editor and type the following code. Save it as {\bf Makefile}.

\lstinputlisting{./codes/Makefile}        




\end{enumerate}
%\section{Hardware Setup}
%\begin{enumerate}
%\item Attach the icoboard to the RaspberryPi through GPIO pins.
%\item Boot RaspberryPi from the image written SD card.
%\item Open a terminal window. Run the following commands:
%\begin{verbatim}

%git clone https://github.com/
%cliffordwolf/icotools

%cd icotools/examples/blinky

%make prog_sram

%\end{verbatim}

%\item Now all three LEDs on icoboard perform a slow blink pattern.


%\end{enumerate}
\section{Example Codes}


\subsection{Blink LED}

\begin{enumerate}
    \item In the same directory, open a text editor and type the following verilog code and save it as \textbf{blink.v}
\lstinputlisting{./codes/blinky/blink.v}        

    \item In addition to the verilog file, we need to indicate to which FPGA pin we want to connect the A output.
    \item This mapping is done in the file .pcf (pcf = Physical Constraint file).
    \item Open a text editor and type the following and save it as \textbf{blink.pcf}
\lstinputlisting{./codes/blinky/blink.pcf}        
\item Setup the seven segment display on the breadboard.
\item Make pin connections according to Table \ref{table:blink_pin}.
\input{./figs/blink_pin.tex}
%   \item Now connect B7 pin of icoboard to positive end of LED.
%   \item Connect GND pin of icoboard to negative end of LED.
    
%\item Now open the {\bf Makefile}. 
%Replace V\_FNAME with Blink\_LED, and PCF\_File with Blink    \_LED.  
   % \item Go to the given link which have all commands to run verilog code.
    
   % \url{http://tlc.iith.ac.in/img/icoboard/Makefile}
   % \item Copy everything into a text editor and save it as "Makefile".
   % \item Open the Makefile, and change all words with "example" to "Blink\_LED"
    
    \item Now open terminal and go to the Directory where all files are saved and type the following command.
    \begin{verbatim}
make v_fname=blink 
    \end{verbatim}
    \item What do you observe?
\item Find out how the delay is obtained in the above code, given that the clock frequency of the FPGA is 100 MHz.    
    \item Modify \textbf{blink.v} to turn the LED on and Off
\end{enumerate}

\subsection{Display}
\begin{enumerate}
\item    Make pin connections from the Icoboard to the seven segment display in Fig. \ref{fig:sevenseg} according to Table \ref{table:2}.

\begin{figure}[!h]
\begin{center}
\resizebox {0.5\columnwidth} {!} {
\input{./figs/sevenseg.tex}
}
\end{center}
\caption{}
\label{fig:sevenseg}
\end{figure}

\begin{table}   
\begin{center}
\begin{tabular}{ | m{3em} | m{2cm}| } 
\hline
Pin & Segment   \\\hline
A5 & a   \\ \hline
A2 & b    \\\hline
C3 & c  \\\hline
B4 & d\\\hline
B7 & e  \\\hline
B6 & f  \\\hline
B3 & g  \\\hline
Vcc & COM \\\hline

\end{tabular}
\end{center}
\caption{Pin Connections}
\label{table:2}

\end{table}

\item Execute the following verilog code
\lstinputlisting{./codes/sevenseg/sevenseg.v}    
with  pcf
\lstinputlisting{./codes/sevenseg/sevenseg.pcf}    
%
What do you observe?
\item Use the above codes to generate numbers from 0-9 on the display. Comment.
\end{enumerate}
\subsection{Decade Counter}

\begin{enumerate}
    \item Execute the following verilog code

\lstinputlisting{./codes/display_decoder/display_decoder.v}    
with  pcf
    \lstinputlisting{./codes/display_decoder/display_decoder.pcf}        
and observe the output on the display.


%   \item Now connect icoboard to sevensegment display in Fig. \ref{fig:sevenseg}  according to Table \ref{table:2}.
   % \item Go to the given link which have all commands to run verilog code.
    
    %\url{http://tlc.iith.ac.in/img/icoboard/Makefile}
    %\item Copy everything into a text editor and save it as "Makefile".
%    \item Copy the Makefile into the same directory and replace V\_FNAME with decade\_counter, and PCF\_File with decade\_counter.
    
%    \item Now open terminal and go to the Directory where all files are saved and type the following command.
%    \begin{verbatim}
%make
%    \end{verbatim}
%    \item Execute the program using make.

\end{enumerate}
 

\end{document}

